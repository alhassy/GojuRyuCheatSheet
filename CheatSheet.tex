% Created 2019-03-04 Mon 16:23
% Intended LaTeX compiler: pdflatex
\documentclass[11pt]{article}
\usepackage[utf8]{inputenc}
\usepackage[T1]{fontenc}
\usepackage{graphicx}
\usepackage{grffile}
\usepackage{longtable}
\usepackage{wrapfig}
\usepackage{rotating}
\usepackage[normalem]{ulem}
\usepackage{amsmath}
\usepackage{textcomp}
\usepackage{amssymb}
\usepackage{capt-of}
\usepackage{hyperref}
\usepackage{minted}
\usepackage{titling}
\def\maketitle#1{}
\usepackage[landscape,twocolumn, margin=0.5in]{geometry}
\usepackage{eufrak} % for mathfrak fonts
\usepackage{multicol}
\usepackage[dvipsnames]{xcolor} % named colours
\usepackage{color}
\definecolor{darkgreen}{rgb}{0.0, 0.3, 0.1}
\definecolor{darkblue}{rgb}{0.0, 0.1, 0.3}
\hypersetup{colorlinks,linkcolor=darkblue,citecolor=darkblue,urlcolor=darkgreen}
\setlength{\parindent}{0pt}
\usepackage{enumitem}
\RequirePackage{fancyvrb}
\DefineVerbatimEnvironment{verbatim}{Verbatim}{fontsize=\scriptsize}
\author{\href{http://www.cas.mcmaster.ca/\~alhassm/}{Musa Al-hassy} {\tiny\hspace{6em}\url{ https://github.com/alhassy/GojuRyuCheatSheet } }}
\date{\today}
\title{Goju-Ryu Karate\\\medskip
\large Cheat Sheet}
\hypersetup{
 pdfauthor={\href{http://www.cas.mcmaster.ca/\~alhassm/}{Musa Al-hassy} {\tiny\hspace{6em}\url{ https://github.com/alhassy/GojuRyuCheatSheet } }},
 pdftitle={Goju-Ryu Karate},
 pdfkeywords={},
 pdfsubject={This document is written by Musa Al-hassy for his learning in the Winter of 2019.},
 pdfcreator={Emacs 26.1 (Org mode 9.2.1)}, 
 pdflang={English}}
\begin{document}

\maketitle

\fontsize{9}{10}\selectfont

\theauthor \hfill \thedate
\hline
{\center \large\bf \thetitle \\ }


\definecolor{grey}{rgb}{0.5,0.5,0.5}


%
% Note the * makes the numbering dissapear; 
% See §7.2 of the manual: http://mirror.its.dal.ca/ctan/macros/latex/base/classes.pdf
% 
\makeatletter
\renewcommand\section[1]{
  \@startsection {section}{1}{0ex}%
                 {-3.5ex \@plus -1ex \@minus -.2ex}%
                 {-1em}%
		 { \color{black}\normalfont\bfseries}* {\fbox{#1} \vspace{1ex}\newline }}
		 
\makeatother

% The black-colour is to ensure no accidental spill-over when using other colour; e.g. \invisibleHI

\def\labelitemi{$\diamond$}
\def\labelitemii{$\circ$}
\def\labelitemiii{$\star$}

% Level 0                 Level 0
% + Level 1               ⋄ Level 1 
%   - Level 2       --->      ∘ Level 2 
%     * Level 3                   ⋆ Level 3
% 

\setitemize{itemsep=2pt,topsep=0pt,parsep=0pt,partopsep=0pt}
\setdescription{itemsep=0.3em,topsep=0pt,parsep=0pt,partopsep=0pt}

\renewenvironment{parallel}[1][2] % one argument, whose default value is literal `2`.
 {
  \setlength{\columnseprule}{2pt}
  \begin{minipage}[t]{\linewidth} % width of minipage is 100% times of \linewidth
  \begin{multicols}{#1}  % default is `2`
 }
 { 
  \end{multicols}
  \end{minipage}
 }

% Common case is to have three columns, want to avoid invoking the attribute via org, so making this.
\newenvironment{parallel3}
 {
  \setlength{\columnseprule}{2pt}
  \begin{minipage}[t]{\linewidth} % width of minipage is 100% times of \linewidth
  \begin{multicols}{3}
 }
 { 
  \end{multicols}
  \end{minipage}
 }


% paralellNB, this is paralell enviro but with `N`o  `B`ar in-between the columns.

\newenvironment{parallelNB}[1][2] % one argument, whose default value is literal `2`.
 {
  \setlength{\columnseprule}{0pt}
  \begin{minipage}[t]{\linewidth} % width of minipage is 100% times of \linewidth
  \begin{multicols}{#1}  % default is `2`
 }
 { 
  \end{multicols}
  \end{minipage}
 }

\newenvironment{parallel3NB}
 {
  \setlength{\columnseprule}{0pt}
  \begin{minipage}[t]{\linewidth} % width of minipage is 100% times of \linewidth
  \begin{multicols}{3}
 }
 { 
  \end{multicols}
  \end{minipage}
 }

\def\providedS{ \qquad\Leftarrow\qquad }

\def\impliesS{ \qquad\Rightarrow\qquad }

\def\landS{ \qquad\land\qquad }
\def\lands{ \quad\land\quad }

\def\eqs{ \quad=\quad}

\def\equivs{ \quad\equiv\quad}
\def\equivS{ \qquad\equiv\qquad}

\def\begineqns{ \begingroup\setlength{\abovedisplayskip}{-1pt}\setlength{\belowdisplayskip}{-1pt} }
\def\endeqns{ \endgroup }

\def\room{\vspace{0.5em}}

% Usage: \eqn{name}{formula}
\setlength{\abovedisplayskip}{5pt} \setlength{\belowdisplayskip}{2pt}
\def\eqn#1#2{ \begin{flalign*} #2 && \tag*{\sc #1} \label{#1} \end{flalign*}  }

% \setlength{\parskip}{1em}


\def\invisibleHI{ \invisible{Hi} }
\def\invisible#1{ {\color{white}{#1}}  }

\def\forcenewline{ {\color{white} .} \newline }
\def\forcenewpage{ {\color{white} .} \newpage }

\section{General Terms}
\label{sec:orgd2e0074}
\begin{center}
\begin{tabular}{ll}
Sensei & teacher\\
O'Sensei & Teacher's teacher\\
Dojo & training hall\\
Nippon & Japan\\
Kihon & Basics\\
Gi & Uniform\\
Obi & Belt\\
Kyu & Boy\\
Dan & Man\\
Mukso & Meditate\\
Kimi & Focus\\
Kilskai & Attention\\
Rei & Bow\\
Ashimai & Begin\\
Yemai & Stop\\
Kia & Yell\\
\end{tabular}
\end{center}

\section{Dachi --Stance}
\label{sec:org0eb2c4f}
\begin{center}
\emph{Power is rooted in the feet, developed by the knees, and directed by the hips!}
\end{center}

\begin{center}
\begin{tabular}{ll}
Sanchin   dachi & Power   stance\\
Zenkutsu  dachi & Forward stance\\
Sheko     dachi & Straddle stance\\
Heiko     dachi & Natural stance\\
Musubu    dachi & Ready   stance\\
Neko ashi dachi & Cat     stance\\
\end{tabular}
\end{center}
\section{Geri --Kick}
\label{sec:org334c71a}
\begin{center}
\emph{Except the instep roundhouse, remember to curl your toes for each and every kick!}
\end{center}
\begin{center}
\begin{tabular}{ll}
Mae Geri & Front kick\\
Kensetsu Geri & Joint kick\\
Mawashi Geri & Roundhouse kick\\
Ushiro Geri & Back kick\\
Yoko Geri & Side kick\\
Fumi Komi Geri & Stomp kick\\
Hiza Geri & Knee kick\\
Mikasuki Geri & Crescent kick\\
\end{tabular}
\end{center}

\begin{center}
\emph{Remember the join sequence: All kicks start with the hips, work to the knee, and finish with the ankle.}
\end{center}
\section{Uke --Block}
\label{sec:org38c5cfd}
\begin{center}
\emph{90\% of blocks are executed with the front hand, making it easier to counter!}
\end{center}

\begin{center}
\begin{tabular}{ll}
Hariatoshi & 3 point low block\\
Kake uke & Hooking block\\
Mawashi uke & Roundhouse block\\
Age uke & High block\\
Uchi uke & Inside block\\
Soto uke & Outside block\\
Gedan bari & Low sweeping block\\
Kakuto uke & Chicken head block\\
Teisho uke & Palm block\\
\end{tabular}
\end{center}

\section{Te --Hand Strikes}
\label{sec:orgc2395a0}
\begin{center}
\emph{The principles of expansion and contraction are a must when striking and as well for blocking!}
\end{center}

\begin{center}
\begin{tabular}{ll}
Oi zuke & Lunge punch\\
Gyaku zuke & Reverse punch\\
Kizama zuke & Jab punch\\
Shuto & Knifehand\\
Teisho & Palm heel\\
Empi & Elbow\\
Rekkan & Backfist\\
Nukite & Finger strike\\
Mawashi zuke & Hook punch\\
\end{tabular}
\end{center}

\section{Japanese Numbers}
\label{sec:org0affc43}
\begin{center}
\begin{tabular}{rl}
0 & rei\\
1 & ichi\\
2 & ni\\
3 & san\\
4 & yon\\
5 & go\\
6 & roku\\
7 & nana\\
8 & hachi\\
9 & kyu\\
10 & juu\\
\end{tabular}
\end{center}

\newpage
\section{Kata  --Forms}
\label{sec:org8140f14}
\begin{center}
\begin{tabular}{ll}
Taikyoku Gedan & First course lower\\
Taikyoku Chudan & First course middle\\
Taikyoku Jodan & First course upper\\
Taikyoku Mawashi Uke & First course circular block\\
Taikyoku Kake Uke & First course hooking block\\
Gekisai Itch & Attack \& Smash 1\\
Gakisai Ni & Attack \& Smash 2\\
Sanchin & Three battles\\
Tensho & Turning\\
Saifa & Destroy, Defeat\\
Seienchin & Attack, Conquer, Suppress\\
Sanseiru & Thirty-six movements\\
Shi Sho Chin & Twenty-seven movements\\
Seisun & Fifty-six movements\\
Seipai & Eighteen movements\\
\end{tabular}
\end{center}
\end{document}
