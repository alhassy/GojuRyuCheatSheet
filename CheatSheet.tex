% Created 2019-03-24 Sun 12:05
% Intended LaTeX compiler: pdflatex
\documentclass[11pt]{article}
\usepackage[utf8]{inputenc}
\usepackage[T1]{fontenc}
\usepackage{graphicx}
\usepackage{grffile}
\usepackage{longtable}
\usepackage{wrapfig}
\usepackage{rotating}
\usepackage[normalem]{ulem}
\usepackage{amsmath}
\usepackage{textcomp}
\usepackage{amssymb}
\usepackage{capt-of}
\usepackage{hyperref}
\usepackage{minted}
\usepackage{titling}
\def\maketitle#1{}
\usepackage[landscape,twocolumn, margin=0.5in]{geometry}
\usepackage{eufrak} % for mathfrak fonts
\usepackage{multicol}
\usepackage[dvipsnames]{xcolor} % named colours
\usepackage{color}
\definecolor{darkgreen}{rgb}{0.0, 0.3, 0.1}
\definecolor{darkblue}{rgb}{0.0, 0.1, 0.3}
\hypersetup{colorlinks,linkcolor=darkblue,citecolor=darkblue,urlcolor=darkgreen}
\setlength{\parindent}{0pt}
\usepackage{enumitem}
\RequirePackage{fancyvrb}
\DefineVerbatimEnvironment{verbatim}{Verbatim}{fontsize=\scriptsize}
\author{The McMaster Dojo {\tiny\hspace{6em}\url{ https://github.com/alhassy/GojuRyuCheatSheet } }}
\date{\today}
\title{Goju-Ryu Karate\\\medskip
\large Cheat Sheet}
\hypersetup{
 pdfauthor={The McMaster Dojo {\tiny\hspace{6em}\url{ https://github.com/alhassy/GojuRyuCheatSheet } }},
 pdftitle={Goju-Ryu Karate},
 pdfkeywords={},
 pdfsubject={This document is written by Musa Al-hassy for his learning in the Winter of 2019.},
 pdfcreator={Emacs 26.1 (Org mode 9.2.2)}, 
 pdflang={English}}
\begin{document}

\maketitle

\fontsize{9}{10}\selectfont

\theauthor \hfill \thedate
\hline
{\center \large\bf \thetitle \\ }


\definecolor{grey}{rgb}{0.5,0.5,0.5}


%
% Note the * makes the numbering dissapear; 
% See §7.2 of the manual: http://mirror.its.dal.ca/ctan/macros/latex/base/classes.pdf
% 
\makeatletter
\renewcommand\section[1]{
  \@startsection {section}{1}{0ex}%
                 {-3.5ex \@plus -1ex \@minus -.2ex}%
                 {-1em}%
		 { \color{black}\normalfont\bfseries}* {\fbox{#1} \vspace{1ex}\newline }}
		 
\makeatother

% The black-colour is to ensure no accidental spill-over when using other colour; e.g. \invisibleHI

\def\labelitemi{$\diamond$}
\def\labelitemii{$\circ$}
\def\labelitemiii{$\star$}

% Level 0                 Level 0
% + Level 1               ⋄ Level 1 
%   - Level 2       --->      ∘ Level 2 
%     * Level 3                   ⋆ Level 3
% 

\setitemize{itemsep=2pt,topsep=0pt,parsep=0pt,partopsep=0pt}
\setdescription{itemsep=0.3em,topsep=0pt,parsep=0pt,partopsep=0pt}

\renewenvironment{parallel}[1][2] % one argument, whose default value is literal `2`.
 {
  \setlength{\columnseprule}{2pt}
  \begin{minipage}[t]{\linewidth} % width of minipage is 100% times of \linewidth
  \begin{multicols}{#1}  % default is `2`
 }
 { 
  \end{multicols}
  \end{minipage}
 }

% Common case is to have three columns, want to avoid invoking the attribute via org, so making this.
\newenvironment{parallel3}
 {
  \setlength{\columnseprule}{2pt}
  \begin{minipage}[t]{\linewidth} % width of minipage is 100% times of \linewidth
  \begin{multicols}{3}
 }
 { 
  \end{multicols}
  \end{minipage}
 }


% paralellNB, this is paralell enviro but with `N`o  `B`ar in-between the columns.

\newenvironment{parallelNB}[1][2] % one argument, whose default value is literal `2`.
 {
  \setlength{\columnseprule}{0pt}
  \begin{minipage}[t]{\linewidth} % width of minipage is 100% times of \linewidth
  \begin{multicols}{#1}  % default is `2`
 }
 { 
  \end{multicols}
  \end{minipage}
 }

\newenvironment{parallel3NB}
 {
  \setlength{\columnseprule}{0pt}
  \begin{minipage}[t]{\linewidth} % width of minipage is 100% times of \linewidth
  \begin{multicols}{3}
 }
 { 
  \end{multicols}
  \end{minipage}
 }

\def\providedS{ \qquad\Leftarrow\qquad }

\def\impliesS{ \qquad\Rightarrow\qquad }

\def\landS{ \qquad\land\qquad }
\def\lands{ \quad\land\quad }

\def\eqs{ \quad=\quad}

\def\equivs{ \quad\equiv\quad}
\def\equivS{ \qquad\equiv\qquad}

\def\begineqns{ \begingroup\setlength{\abovedisplayskip}{-1pt}\setlength{\belowdisplayskip}{-1pt} }
\def\endeqns{ \endgroup }

\def\room{\vspace{0.5em}}

% Usage: \eqn{name}{formula}
\setlength{\abovedisplayskip}{5pt} \setlength{\belowdisplayskip}{2pt}
\def\eqn#1#2{ \begin{flalign*} #2 && \tag*{\sc #1} \label{#1} \end{flalign*}  }

% \setlength{\parskip}{1em}


\def\invisibleHI{ \invisible{Hi} }
\def\invisible#1{ {\color{white}{#1}}  }

\def\forcenewline{ {\color{white} .} \newline }
\def\forcenewpage{ {\color{white} .} \newpage }

\section{Dachi --Stance}
\label{sec:orge339ad6}
\begin{center}
\emph{Power is rooted in the feet, developed by the knees, and directed by the hips!}
\end{center}

\begin{center}
\begin{tabular}{ll}
Heiko     dachi & Natural stance\\
Heisoku   dachi & Attention stance\\
Zenkutsu  dachi & Forward stance\\
Shiko     dachi & Horse, Straddle stance\\
Sanchin   dachi & Hour Glass, Power   stance\\
Neko ashi dachi & Cat stance\\
Tsuru ashi dachi & Stork stance\\
Teiji dachi & t-stance\\
Kokutsu dachi & back stance\\
\end{tabular}
\end{center}
\section{Uke --Block}
\label{sec:orgad6aae4}
\begin{center}
\emph{90\% of blocks are executed with the front hand, making it easier to counter!}
\end{center}

\begin{center}
\begin{tabular}{ll}
Gedan bari & Low sweeping block\\
Jodan uke & Rising block\\
Chudan uke & Inside forearm block\\
Soto uke & Outside forearm block\\
Hariatoshi & Low 3 point block\\
Kakewaki uke & Cross block\\
Mawashi uke & Roundhouse block\\
Kake uke & Hook block\\
Hiza uke & Shin block\\
Kakuto uke & Crane head block\\
Shuto uke & Knife hand block\\
Morto uke & Augmented forearm block\\
Teisho uke & Palm block\\
\end{tabular}
\end{center}

\newpage
\section{Te --Hand Strikes}
\label{sec:orgde7aef0}
\begin{center}
\vspace{-1em}
\emph{The principles of expansion and contraction are a must when striking and as well for blocking!}
\end{center}

\begin{center}
\begin{tabular}{ll}
Oi zuki & Lunge punch\\
Gy Aku zuki & Reverse punch\\
Kizama zuki & Front jab\\
Morto zuki & Double fist punch\\
Rek Ken & Backfist\\
Ura Ken & Back fist\\
Tetsui & Hammer fist\\
Shuto & Knifehand\\
Teisho & Palm heel\\
Empi & Elbow\\
Mawashi zuki & Hook punch\\
Nukite & Finger strike\\
\end{tabular}
\end{center}

\section{Geri --Kick}
\label{sec:org628e439}

\begin{center}
\vspace{-1em}
\emph{Except the instep roundhouse, remember to curl your toes for each and every kick!}
\end{center}

\begin{center}
\begin{tabular}{ll}
Mae Geri Kekomi & Front thrust kick\\
Mae Geri Keage & Front snap kick\\
Yoko Geri Kekomi & Side thrust kick\\
Yoko Geri Keage & Side snap kick\\
Kinsetsu Geri & Joint kick, Knee break\\
Mawashi Geri & Roundhouse kick\\
Fumi Komi Geri & Stomp kick\\
Ushiro Geri & Back kick\\
Ashi Barai & Foot sweep\\
Mikasuki Geri & Crescent kick\\
\end{tabular}
\end{center}

\begin{center}
\vspace{-0.3em}
\emph{Remember the joint sequence: All kicks start with the hips, work to the knee, and finish with the ankle.}
\end{center}
\section{Kata  --Forms}
\label{sec:org4420240}
A \emph{kata} is the execution of a pre-defined series of movements which simulates
a confrontation against multiple opponents.
As such, remember to place yourself in the midst of your enemies and to visualise them
--this in turn necessitates rhythm and timing.
Moreover, full all-out intense kata will quickly demonstrate the importance of breathing.

\room
The key points of kata are:
\begin{center}
\begin{tabular}{llllll}
Pace & Breathing & Eyes & Focus & Kiai & Technique\\
\end{tabular}
\end{center}

\begin{center}
\emph{Always concentrate on posture. Never rush through the kata.}
\emph{You will become what you practice.} 
\emph{If you rush your movements without purpose, your kata will become ragged and weak.}
\end{center}

\newpage
\begin{center}
\begin{tabular}{l}
\textbf{Kata --Forms}\\
\end{tabular}
\end{center}

\begin{center}
\begin{tabular}{ll}
Taikyoku Gedan & First course lower\\
Taikyoku Chudan & First course middle\\
Taikyoku Jodan & First course upper\\
Taikyoku Mawashi Uke & First course circular block\\
Taikyoku Kake Uke & First course hooking block\\
Gekisai Itch & Attack \& Smash 1\\
Gakisai Ni & Attack \& Smash 2\\
Sanchin & Three Battles\\
Tensho & Rotating Palm\\
Saifa & Destroy \& Defeat\\
Seienchin & Attack, Conquer, Suppress\\
San Sei Ryu & Thirty-six movements\\
Shi Sho Chin & Twenty-seven movements\\
Seisan & Fifty-six movements\\
Seipai & Eighteen movements\\
\end{tabular}
\end{center}

\section{Goju History}
\label{sec:org9d1b9df}

\begin{itemize}
\item Sixth century Indian Buddhist monk Bodhidharma immigrated to China is credited with being the father of the martial arts.
\item Okinawa is a large island between Japan and China where many Chinese immigrated for trade.
\item As a major trading post, Okinawa produced a composite fighting style known as \emph{Te}, “hand”, a forerunner of \emph{karate}, “empty hand”.
\item In 1904, it was introduced into Okinawan public schools as part of the physical education curriculum.
\item In 1915, Okinawan Gichin Funakoshi brought the art to Japan, founding shotokan karate.
\begin{itemize}
\item ‘Shoto’ is pine-waves and ‘kan’ is hall; shotokan was the name of his dojo.
\end{itemize}
\item Okinawan Miyagi Chojun popularised karate and incorporated Chinese forms to produce \emph{goju ryu}, “hard soft style”.
\begin{itemize}
\item He revised sanchin –the hard aspect of goju– and created tensho– the soft aspect.
\item These two kata are considered the essence of Goju Ryu.
\item The highest kata, Suparinpei, is said to contain the full syllabus of Goju Ryu.
\end{itemize}

He designed goju's official insignia: The goju fist, which is half open and half closed preserving the
dichotomy that goju ryu is neither completely hard nor completely soft.
\begin{itemize}
\item The characters on the wrist read \emph{go ju ryu}, while those on the banner read \emph{kara te do}.
\item Its colours are white symbolising the purity of the beginner, black associated with the ideals of being
a black belt, and red for mastery of the artform –the more red shaded in, the higher the rank of the bearer.
\end{itemize}

\item Miyagi's student Gogen ‘the cat’ Yamaguchi formed the All-Japan Karate-do Goju-kai, spreading the style throughout the world.
\end{itemize}

\begin{center}
\emph{The ultimate aim of karate lies not in victory or defeat but in the perfection of character of its participants.}
–Gichin Funakoshi
\end{center}

\newpage

\section{Local Goju History}
\label{sec:org47cf233}
\begin{description}
\item[{Don Warrener, 6ᵗʰ Dan}] Popularised Goju across southern Ontario.
\begin{itemize}
\item Practising Goju since the age of 17; won the Canadian Karate Championships in 1968.
\item Studied under Richard Kim, Bob Dalgleish, Frank Lee, and 
Goshi Yamaguchi --Gogen's son.
\end{itemize}

\item[{Philip McColl, 5ᵗʰ Dan}] Sensei of McMaster Dojo since 1984
--the dojo was founded by Ray Greenway in 1969.
\begin{itemize}
\item Began Karate at the age of 24 at the Delta club in Hamilton, run by Don Warrener.
\item Three of his graduated Shodans are World Champions
\item Proficient in Goju Ryu, Kobudo --weaponry--, and Jujitsu.
\item Student Chris Stafford is the founder and Sensei of the \href{http://torontokarate.ca/}{Toronto Goju Ryu dojo}.
\end{itemize}
\end{description}

\begin{center}
\emph{The ultimate aim of karate lies not in victory or defeat,}
\emph{but in the perfection of character of its participants.}
─Gichin Funakoshi
\end{center}

\section{General Terms}
\label{sec:org4a243ce}

\begin{parallel}


\begin{center}
\begin{tabular}{ll}
Sensei & teacher\\
O'Sensei & Teacher's teacher\\
Dojo & training hall\\
Nippon & Japan\\
Kihon & Basics\\
Gi & Uniform\\
Obi & Belt\\
Kyu & Boy\\
Dan & Man\\
Mukso & Meditate\\
Kimi & Focus\\
Kilskai & Attention\\
Rei & Bow\\
Ashimai & Begin\\
Yemai & Stop\\
Kia & Yell\\
\end{tabular}
\end{center}

\begin{center}
\begin{tabular}{l}
\href{https://www.linguajunkie.com/japanese/count-in-japanese-1-100}{\uline{Japanese Numbers}}\\
\end{tabular}
\end{center}

\begin{center}
\begin{tabular}{rl}
0 & rei\\
1 & ichi\\
2 & ni\\
3 & san\\
4 & yon\\
5 & go\\
6 & roku\\
7 & nana\\
8 & hachi\\
9 & kyuu\\
10 & juu\\
100 & hyaku\\
1000 & sen\\
\end{tabular}
\end{center}
\end{parallel}

\begin{itemize}
\item 11, 12, …, 19  are juu ichi, juu ni, …, juu kyuu.
\item 20, 30, …, 90  are ni juu, san juu, …, kyuu juu.
\begin{itemize}
\item Likewise, 200, …, 900 are ni hyaku, …, kyuu hyaku.
\begin{itemize}
\item Exceptions: 300 san byaku, 600 ro ppyaku, 800 ha ppyaku.
\end{itemize}
\end{itemize}
\item twenty-one, sixty-nine, eighty-two, … are 
ni juu ichi, roku juu kyuu, hachi juu ni, … .
\begin{itemize}
\item Likewise, 368 is three hundred and sixty-eight which is san byaku roku juu hachi.
\end{itemize}
\end{itemize}
\end{document}
